\documentclass{article}
\usepackage{graphicx}
\usepackage{amsmath}
\usepackage[utf8]{inputenc}
\usepackage[left=0.00cm,right=0.00cm,top=0.00cm,bottom=0.00cm,includefoot, headheight=13.6pt]{geometry}
\usepackage{hyperref}
\hypersetup{colorlinks=true, linkcolor=blue, filecolor=magenta, urlcolor=cyan, pdftitle={Original}, pdfauthor={PDF Conversion Service}}
\title{Original}
\author{PDF Conversion Service}
\date{\today}
\begin{document}
\maketitle
Introduction \\ L'accroissement des capacit´es de simulation num´erique a fait ´emerger de nouvelles probl´ema- \\ tiques dans le spectre de l'ing´enieur. Apr`es 30 ans de d´eveloppement en mod´elisation \\ num´erique, les probabilit´es fournissent un cadre adapt´e `a la prise en compte des incertitudes \\ inh´erentes aux erreurs de mod´elisation et `a la variabilit´e de l'environnement. \\ Sur la base de la th´eorie moderne des probabilit´es, les processus al´eatoires permettent de \\ mod´eliser des ph´enom`enes fluctuants ou stochastiques dont l'´evolution dans le temps revˆet \\ une grande importance. Par exemple, l'ajout de termes al´eatoires dans les EDP r´egissant \\ les ph´enom`enes physiques, tels que la turbulence, conduit `a des solutions qui sont des \\ processus stochastiques. Dans ce cadre, les d´ependances avec le pass´e peuvent ˆetre d´ecrites \\ par des fonctions de corr´elation (processus gaussiens), par leurs esp´erances conditionnelles \\ (martingales) ou par leurs lois conditionnelles (processus de Markov). \\ Pour maitriser ces concepts modernes, il est n´ecessaire de bien comprendre le cas o`u les \\ processus sont index´es par des ´el´ements d'ensembles discrets avant d'aborder les outils \\ complexes issus du calcul stochastique intervenant en conception robuste, mod´elisation \\ physique et financi`ere ou traitement du signal et d'images. \\ Avertissement : \\ Ce document n'a pas pour objectif de constituer un support de cours \\ complet. Il s'apparente `a un plan d´etaill´e contenant les renvois pr´ecis `a des r´ef´erences \\ accessibles sur Internet. \\ La premi`ere partie de ce cours s'appuie largement sur les r´ef´erences [6, 13, 11]. \\ 4 \\ Rappels d'esp´erance conditionnelle \\ R´ef´erence : Lecture Notes du cours CIP de 1`ere ann´ee ([5]). \\ Dans l'espace de probabilit´e (Ω, F, P), si A et B sont dans F et P(B) > 0, la probabilit´e \\ conditionnelle de A sachant B est d´efinie par \\ P(A | B) = P(A ∩B) \\ P(B) \\ . \\ Si X et Y sont des variables al´eatoires `a valeurs respectivement dans R et dans un espace \\ discret E. Il est int´eressant de connaitre la distribution ou l'esp´erance de X, connaissant la \\ r´ealisation de Y . Pour tout y ∈E, on a \{Y = y\} ∈F donc si P(Y = y) > 0, alors on peut \\ consid´erer la mesure de probabilit´e \\ Q : F →[0, 1] \\ A 7→P(A | Y = y). \\ Si X ∈L1(Ω, F, Q), l'esp´erance conditionnelle de X sachant \{Y = y\} est d´efinie par \\ E[X | Y = y] = EQ[X], \\ o`u EQ[X] d´esigne l'esp´erance de X sous la probabilit´e Q. \\ Par exemple, dans le cas o`u X est ´egalement `a valeurs dans un espace discret ˜E, \\ E[X | Y = y] = \\ X \\ x∈˜E \\ x P(X = x | Y = y), \\ si la somme est convergente. \\ Pour d´efinir l'esp´erance de X sachant la variable al´eatoire Y , on consid`ere \\ ψ : y 7→ \\  E[X | Y = y] \\ si P(Y = y) > 0 \\ 0 \\ sinon. \\ L'esp´erance de X sachant Y est alors d´efinie par \\ E[X | Y ] = ψ(Y ). \\ La construction moderne d´ecrite dans ce chapitre permet d'´etendre la d´efinition de \\ l'esp´erance conditionnelle de X sachant une variable al´eatoire Y qui n'est pas n´ecessairement \\ discr`ete, et pour laquelle les ´ev´enements \{Y = y\} sont de probabilit´e nulle. \\ 5 \\ 


\end{document}